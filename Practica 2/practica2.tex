\documentclass{article}

\usepackage[spanish]{babel}
\usepackage[numbers,sort&compress]{natbib}
\usepackage{graphicx}
\usepackage{url}
\usepackage{amsmath}
\usepackage{hyperref}
\usepackage[top=30mm, bottom=40mm, left=15mm, right=15mm]{geometry}
\setlength{\parskip}{2mm}
\setlength{\parindent}{0pt}

\author{Abraham Azael Morales Juárez 1422745}
\title{Autómata celular}
\date{\today}

\begin{document}

\maketitle

\section{Introducción}
En esta práctica de autómatas celulares, que serán representadas por una matriz booleana (que es una matriz que contiene 0 y 1 los cuales representan la vida y la muerte, respectivamente), determinaremos la supervivencia de cada celda con base de los valores de sus ochos vecinos \cite{Practica}.

\section{Objetivo}
Determinar el número de iteraciones que procede de la simulación en una malla de 30 por 30 celdas hasta que se mueran todas, variando la probabilidad inicial de celda viva entre cero y uno en pasos de 0.10.
\section{Resultados}

\bibliographystyle{plainnat} 
\bibliography{ref}

\end{document}